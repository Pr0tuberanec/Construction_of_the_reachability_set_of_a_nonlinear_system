\documentclass[a4paper,12pt]{article}
\usepackage[utf8]{inputenc}	
\usepackage[T1, T2A]{fontenc}	
\usepackage{indentfirst}	
\usepackage[english, russian]{babel}	
\usepackage{indentfirst}
\usepackage{a4wide}	
\usepackage{amsmath}
\usepackage{amsthm} 
\usepackage{amssymb}
\usepackage{float}
\usepackage{arcs}
\usepackage[top=2cm, bottom=2cm, left=2.5cm, right=1cm]{geometry}
\usepackage{amsfonts}
\usepackage{graphicx}
\usepackage[unicode, pdftex]{hyperref}
%\usepackage[unicode, pdftex]{hyperref}	
\usepackage{setspace}
%\usepackage{vmargin}
\sloppy
\usepackage{indentfirst} % Красная строка
\uchyph=0 % Запрет переноса слова с прописной буквы

\linespread{1.3}
\newtheorem{definition}{Определение}
\newtheorem{thm}{Теорема}
\newcommand{\hm}[1]{#1\nobreak\discretionary{}{\hbox{\ensuremath{#1}}}{}}

\begin{document}
\newcommand{\sgn}{\mathrm{sgn}}

\begin{titlepage}
\begin{center}
\includegraphics[width=8cm, height=4cm]{msu.eps}
\end{center}

\begin{center}
Московский государственный университет имени М.В. Ломоносова\\
\vspace{0.2cm}
Факультет вычислительной математики и кибернетики\\
\vspace{0.2cm}
Кафедра системного анализа

\vspace{4cm}
{\LARGE Отчёт по практикуму}\\
\vspace{1cm}
{\Huge\bfseries <<Построение множества достижимости нелинейной системы>>}
\end{center}

\vspace{2cm}
\begin{flushright}
\large
\textit{Студент 315 группы}\\
Н.~Ю.~Заварзин\\
\vspace{5mm}
\textit{Руководитель практикума}\\
к.ф.-м.н., доцент П.~A.~Точилин\\
\end{flushright}
\vspace{4.5cm}

\begin{center}
Москва, 2023
\end{center}
\end{titlepage} 

\newpage

\tableofcontents

\newpage
\section{Постановка задачи}
Задано обыкновенное дифференциальное уравнение:
\begin{equation} \label{eq1}
	\ddot{x}(t) + x(t)\cos(x^2(t)) + 2x(t)\dot{x}(t) = u(t),
\end{equation}

где $x(t) \in \mathbb{R}$, а на возможные значения управляющего параметра $u(t)$ наложено ограничение: $u(t) \in [-\alpha, \alpha]$, $\alpha > 0.$ Задан начальный момент времени $t_0 = 0$ и начальная позиция $x(t_0)=0$, $\dot{x}(t_0) = 0$. Необходимо построить множество достижимости $X(t_1, t_0, x(t_0), \dot{x}(t_0))$ (множество пар $(x(t_1), \dot{x}(t_1))$) в классе программных управлений в заданный момент времени $t_1 \geqslant t_0$, а также исследовать его свойства. 

\section{Аналитическое решение задачи}
Перейдём от дифференциального уравнения второго порядка \eqref{eq1} к системе дифференциальных уравнений первого порядка, для этого сделаем замену $x_1(t) = x(t)$, $x_2(t) = \dot{x}(t)$, тогда
\begin{equation} \label{eq2}
	\begin{cases}
		\dot{x}_1(t) = x_2(t), \\
		\dot{x}_2(t) = -x_1(t)\cos\left(x_1^2(t)\right) - 2x_1(t)x_2(t) + u(t), \\
		x_1(t_0) = 0, \\
		x_2(t_0) = 0.
	\end{cases}
\end{equation}

\subsection{Принцип максимума Понтрягина для задачи достижимости}
Рассмотрим систему в $\mathbb{R}^n$:
\[ \dot{x} = f(x, u),\]
где $f(x, u)$, $\frac{\partial f}{\partial x}(x, u)$~--- непрерывные в $\mathbb{R}^{n+m}$ функции. Пусть $\mathcal{P}$~--- множество допустимых управлений на $t \in [0, T]$ и задано $x(t_0) = x_0$. Пусть некоторому допустимому управлению $u^*(t) \in \mathcal{P}$ соответствует траектория $x^*(t)$ с концом $x^*(T)$, принадлежащем границе множества достижимости, а $\mathcal{H}(\psi, x, u)$~--- функция Гамильтона-Понтрягина рассматриваемых дифференциальных уравнений. Тогда на $[0, T]$ существует ненулевое решение $\psi(t)$ сопряжённой системы
\begin{equation}\label{eq5}
\dot{\psi}(t) = - \dfrac{\partial{\mathcal{H}(\psi, x^*, u^*)}}{\partial{x}},
\end{equation}
такое, что выполняется условие максимума  
\begin{equation}\label{eq4} 
\mathcal{H}(\psi, x^*, u^*) = \sup_{u(\cdot) \ \in \ \mathcal{P}}{\mathcal{H}(\psi, x^*, u)}.
\end{equation} 
Если $u^*(t)$~--- ограничено, то для $t \in [0, T]$ выполнено:
\begin{equation}\label{eq7}
	\sup\limits_{u(\cdot) \ \in \ \mathcal{P}}{\mathcal{H}(\psi, x^*, u)} = \mathrm{const} \geqslant 0.
\end{equation} 

Для построения множества достижимости достаточно найти концы всех траекторий, удовлетворяющих принципу максимума, а потом удалить из них точки, не принадлежащие границе. 
Функция Гамильтона-Понтрягина для \eqref{eq2} имеет вид:
\[ \mathcal{H}(\psi, x, u) = \psi_1x_2 + \psi_2(-x_1\cos\left(x_1^2\right) - 2x_1x_2 + u). \]
Сопряжённая система к \eqref{eq2} запишется как
\begin{equation}\label{eq6}
	\begin{cases}
		\dot{\psi}_1(t) = 2x_2(t)\psi_2(t) + \cos \left( x_1^2(t) \right)\psi_2(t) - 2 \psi_2(t) x_1^2(t) \sin \left( x_1^2(t) \right), \\
		\dot{\psi}_2(t) = -\psi_1(t) + 2x_1(t) \psi_2(t). 
	\end{cases}
\end{equation}

\subsection{Теорема о чередовании нулей переменных}
\begin{thm}
Если $\tau_1 < \tau_2$, $\tau_1$, $\tau_2 \in [t_0, t_1]$, то для систем вида 
\begin{equation}\label{eq*}	
\begin{cases}
	\dot{x}_1(t) = x_2(t), \\
	\dot{x}_2(t) = -f(x_1(t), x_2(t)) + \sgn  \ \psi_2, \\
	\dot{\psi}_1(t) = \psi_2(t) \dfrac{\partial f(x_1, x_2)}{\partial x_1}, \\
	\dot{\psi}_2(t) = -\psi_1(t) + \psi_2(t) \dfrac{\partial f(x_1, x_2)}{\partial x_2}
\end{cases}
\end{equation}
с непрерывно дифференцируемой функцией $f(x_1, x_2)$ справедливы следующие утверждения:

\begin{enumerate}
	\item 
\begin{equation}\label{eq8}	
	\begin{cases}
		\psi_2(\tau_1) = \psi_2(\tau_2) = 0, \\
		x_2(\tau_1) = 0
	\end{cases} \Rightarrow \ \ x_2(\tau_2) = 0.
\end{equation}

	\item 
\begin{equation}\label{eq9}	
	\begin{cases}
		\psi_2(\tau_1) = \psi_2(\tau_2) = 0, \\
		x_2(\tau_1) \neq 0
	\end{cases} \Rightarrow \ \ 
	\begin{cases}
		x_2(\tau_2) \neq 0, \\
		\exists \tau \in (\tau_1, \tau_2): x_2(\tau) = 0.
	\end{cases} 
\end{equation}

	\item
\begin{equation}\label{eq10}	
	\begin{cases}
		x_2(\tau_1) = x_2(\tau_2) = 0, \\
		\psi_2(\tau_1) = 0
	\end{cases} \Rightarrow \ \ 
	\psi_2(\tau_2) = 0.
\end{equation}

	\item
\begin{equation}\label{eq11}	
	\begin{cases}
		x_2(\tau_1) = x_2(\tau_2) = 0, \\
		\psi_2(\tau_1) \neq 0
	\end{cases} \Rightarrow \ \ 
	\begin{cases}
		\psi_2(\tau_2) \neq 0, \\
		\exists \tau \in (\tau_1, \tau_2): \psi_2(\tau) = 0.
	\end{cases} 
\end{equation}
\end{enumerate}

\end{thm}

\textbf{Доказательство:} 

1) 
\[ \psi_2(\tau_1) = 0, \ x_2(\tau_1) = 0 \Rightarrow \] 
\[ \Rightarrow \mathcal{H}(\tau_1) = \psi_1(\tau_1) x_2(\tau_1) - \psi_2(\tau_1) f(x_1(\tau_1), x_2(\tau_1)) + | \psi_2(\tau_1)| = 0,\]
поскольку $\mathcal{H} \equiv \texttt{const}, $ то
\[ \mathcal{H}(\tau_2) = \psi_1(\tau_2) x_2(\tau_2) - \psi_2(\tau_2) f(x_1(\tau_2), x_2(\tau_2)) + | \psi_2(\tau_2)| = 0 \Rightarrow \]
\[ \Rightarrow \psi_1(\tau_2)x_2(\tau_2) = 0.\]
Если $\psi_1(\tau_2) = 0$, получаем $\psi(\tau_2) = 0$~--- противоречие с принципом максимума Понтрягина. Тогда $x_2(\tau_2) = 0$. 

2)
\[ \psi_2(\tau_1) = 0, \ x_2(\tau_1) \neq 0, \ \psi_1(\tau_1) \neq 0 \Rightarrow \] 
\[ \Rightarrow \mathcal{H}(\tau_1) = \psi_1(\tau_1) x_2(\tau_1) - \psi_2(\tau_1) f(x_1(\tau_1), x_2(\tau_1)) + | \psi_2(\tau_1)| \neq 0,\]
поскольку $\mathcal{H} \equiv \texttt{const}, $ то
\[ \mathcal{H}(\tau_2) = \psi_1(\tau_2) x_2(\tau_2) - \psi_2(\tau_2) f(x_1(\tau_2), x_2(\tau_2)) + | \psi_2(\tau_2)| \neq 0 \Rightarrow \]
\[ \Rightarrow x_2(\tau_2) \neq 0.\]
Без ограничения общности, положим $\psi_2(t) \neq 0, \ \forall t \in (\tau_1, \tau_2).$ Тогда возможны две ситуации, в каждой из которых $\dot{\psi}_2(\tau_1)\dot{\psi}_2(\tau_2) < 0$
\begin{figure}[h]
\begin{minipage}[h]{0.47\linewidth}
\includegraphics[width=1\linewidth]{First_situation.png}
\caption{Первая ситуация}
\end{minipage}
\hfill
\begin{minipage}[h]{0.47\linewidth}
\includegraphics[width=1\linewidth]{Second.png}
\caption{Вторая ситуация}
\end{minipage}
\end{figure}

\[ \dot{\psi}_2(\tau_1) = -\psi_1(\tau_1) + \psi_2(\tau_1) \dfrac{\partial f}{\partial x_2} < 0 \ \ (> 0), \]
\[ \dot{\psi}_2(\tau_2) = -\psi_1(\tau_2) + \psi_2(\tau_2) \dfrac{\partial f}{\partial x_2} > 0 \ \ (< 0), \]
\[ \psi_2(\tau_1) = \psi_2(\tau_2) = 0 \Rightarrow \]
\[ \dot{\psi}_2(\tau_1)\dot{\psi}_2(\tau_2) = \psi_1(\tau_1) \psi_1(\tau_2) < 0.\]
Из проведённых ранее рассуждений 
\[ \psi_1(\tau_1)x_2(\tau_1) = \psi_1(\tau_2)x_2(\tau_2) \Rightarrow x_2(\tau_1)x_2(\tau_2) < 0.\]
Поскольку $x_2(t)$~--- непрерывная функция, то $\exists \ \tau \in (\tau_1, \tau_2): x_2(\tau) = 0.$

3)
Рассмотрим функцию:
\[ z(t) = \psi_1(t)x_2(t) + \psi_2(t) \dfrac{\mathrm{d}x_2(t)}{\mathrm{d}t}. \]
$z(t)$~--- кусочно-непрерывна. Так как $\dot{x}_2(t) = -f(x_1(t), x_2(t)) + u(t)$, то разрывы могут быть лишь в моменты переключений. Пусть $t_0$~--- точка непрерывности. Тогда при $t \in U_{\delta}(t_0)$:
\[ \dfrac{\mathrm{d} z(t)}{\mathrm{d} t} = \psi_2(t) \dfrac{\partial f}{\partial x_1}x_2(t) + \psi_1(t) \left(-f(x_1(t), x_2(t)) + u(t) \right) + \left(-\psi_1(t) + \psi_2(t) \dfrac{\partial f}{\partial x_2} \right) \times \]
\[ \times \left(-f(x_1(t), x_2(t)) + u(t)\right) + \psi_2(t) \left( -\dfrac{\partial f}{\partial x_1}x_2(t) - \dfrac{\partial f}{\partial x_2}(-f(x_1(t), x_2(t)) + u(t)) \right) = 0.\]
Следовательно $z(t)$~--- кусочно-постоянная. Если же $t_0$~--- момент переключения, то $\psi_2(t_0) = 0$. Но тогда и
\[ z(t_0 - 0) = z(t_0 + 0) \Rightarrow z(t) \equiv \mathrm{const}. \]
\[ z(\tau_1) = \psi_1(\tau_1)x_2(\tau_1) + \psi_2(\tau_1)\dot{x}_2(\tau_1) = 0, \]
\[ z(\tau_2) = z(\tau_1) = \psi_1(\tau_2)x_2(\tau_2) + \psi_2(\tau_2)\dot{x}_2(\tau_2) = 0. \]
Следовательно $\psi_2(\tau_2) = 0.$  

4) Аналогично п. 3):
\[ z(\tau_1) = \psi_1(\tau_1)x_2(\tau_1) + \psi_2(\tau_1)\dot{x}_2(\tau_1) \neq 0, \]
\[ z(\tau_2) = z(\tau_1) = \psi_1(\tau_2)x_2(\tau_2) + \psi_2(\tau_2)\dot{x}_2(\tau_2) \Rightarrow \psi_2(\tau_2) \neq 0. \]
\[ \dot{x}_2(\tau_1) \dot{x}_2(\tau_2) < 0 \Rightarrow \psi_2(\tau_1)\psi_2(\tau_2) < 0 \Rightarrow \exists \ \tau \in (\tau_1, \tau_2): \psi_2(\tau) = 0.\] 
\hfill
$\qed$

\subsection{Решение исходной задачи}
Оптимальное управление из условия максимума \eqref{eq4} для нашей задачи определяется как 
\begin{equation}\label{eq12}	
	u^*(t) = \begin{cases}
		\alpha \cdot \sgn \ \psi_2, \ \ \ \psi_2(t) \neq 0 \\
		[-\alpha, \alpha], \ \ \ \psi_2(t) = 0.
	\end{cases} 
\end{equation}

При этом в ней невозможен особый режим, так как иначе из формулы для производной $\psi_2(t)$ \eqref{eq*} следует равенство нулю внутри особого режима и $\psi_1(t)$, что противоречит принципу максимума Понтрягина. Значит
 управление почти всюду равно $\alpha \cdot \sgn \ \psi_2$, что позволяет нам пользоваться доказанной ранее теоремой. Отметим также, что число обнулений $\psi_2(t)$ на отрезке $[0, T]$ конечно, ведь в противном случае существует сходящаяся к некоторому $\tilde{t}$ последовательность времён, в точках которой $\psi_2(t) = 0$, откуда из непрерывности $\psi_2(\tilde{t}) = 0$, отсюда в свою очередь из дифференцируемости $\dot{\psi}_2(\tilde{t}) = 0$, что по \eqref{eq6} приводит к $\psi_1(\tilde{t}) = 0$~--- несоответствие с принципом максимума Понтрягина. 

Воспользуемся этим для построения множества достижимости системы. Введём обозначения
$$ \begin{cases}
		\dot{x}_1(t) = x_2(t), \\
		\dot{x}_2(t) = -x_1(t)\cos\left(x_1^2(t)\right) - 2x_1(t)x_2(t) + \alpha, \\
		\dot{\psi}_1(t) = 2x_2(t)\psi_2(t) + \cos \left( x_1^2(t) \right)\psi_2(t) - 2 \psi_2(t) x_1^2(t) \sin \left( x_1^2(t) \right), \\
		\dot{\psi}_2(t) = -\psi_1(t) + 2x_1(t) \psi_2(t). 
\end{cases} \ \ \ (S_{+})$$
$$ \begin{cases}
		\dot{x}_1(t) = x_2(t), \\
		\dot{x}_2(t) = -x_1(t)\cos\left(x_1^2(t)\right) - 2x_1(t)x_2(t) - \alpha, \\
		\dot{\psi}_1(t) = 2x_2(t)\psi_2(t) + \cos \left( x_1^2(t) \right)\psi_2(t) - 2 \psi_2(t) x_1^2(t) \sin \left( x_1^2(t) \right), \\
		\dot{\psi}_2(t) = -\psi_1(t) + 2x_1(t) \psi_2(t). 
\end{cases} \ \ \ (S_{-})$$

Заметим, что в окрестности $\psi_2(t) = 0$, значение $\dot{\psi}_2(t) \approx -\psi_1(t)$ для обеих систем. То есть при переключении с ($S_{+}$) на $(S_{-})$ необходимо, чтобы $\psi_1(\tau^{\text{пер}}) > 0$, а при переключении с $(S_{-})$ на $(S_{+})$ должно выполняться $\psi_1(\tau^{\text{пер}}) < 0$. При этом нормировка вектора $\psi(t)$ в эти самые моменты переключений не повлияет на переменные координат решения, а также на момент следующего обнуления $\psi_2(t)$. В связи с чем удобно принять $\psi_1(\tau^{\text{пер}}) = \pm 1$ в моменты соответствующих переключений.

\subsection{Особые точки}
\begin{definition}
	Точка $(x_1^*, x_2^*)$ называется особой точкой системы \eqref{eq2}, если правая часть системы в этой точке равна 0.
\end{definition}
Приравняем к 0 правые части системы \eqref{eq2}. Получим
$$\begin{cases}
	x_2 = 0, \\
	x_1\cos\left(x_1^2\right) + 2x_1x_2 -u = 0.
\end{cases}$$
Значит неподвижные точки лежат на прямой $x_2 = 0$, а их координаты $x_1$ численно определяются из уравнения $x_1\cos\left(x_1^2\right) = u$.

\section{Приближенное решение задачи}
\subsection{Алгоритм построения решений}
Множество достижимости будем строить по следующему алгоритму:
\begin{enumerate}
	\item Решаем систему $(S_{+})$ (первые её два уравнения) от $t = 0$ с начальными условиями $x_1(0) \hm = x_2(0) = 0$ до первого момента $x_2(\tau) = 0$ или до $t = T$ (тогда принимаем $\tau = T$).
	\item Разбиваем равномерно отрезок $[0, \tau]$ и для каждой точки сетки $t^*$ интегрируем систему $(S_{-})$ с начальными условиями $[x_1(t^*)$, $x_2(t^*)$, $1$, $0]$ пока $t < T$, до $\psi_2(t^{**}) = 0$.
	\item Решаем $(S_{+})$ с начальными $[x_1(t^{**}), x_2(t^{**}), -1, 0]$ до следующего обнуления $\psi_2$.
	\item Далее после каждого чётного переключения интегрируем $(S_{+})$, а после нечётного $(S_{-})$ аналогично пунктам 2 и 3, пока не достигнем рассматриваемой временной границы $T$.
	 \item Тоже самое (пункты 1-4) проделать для старта с $(S_{-})$.
	 \item Строим $\Gamma[T]$~--- множество концов траекторий в момент времени $T$.
	 \item Преобразуем $\Gamma[T]$ в $\partial W[T]$ посредством удаления пересекающихся отрезков. 
	 
	 Для этого будем поочередно фиксировать отрезки из $\Gamma[T]$, строить уравнение прямой $l_{curr}$, содержащей текущий отрезок, и определять какие из оставшихся сегментов имеют точки, лежащие по разные стороны от $l_{curr}$. Каждый из таких отрезков аналогично задает прямую $l$. Затем необходимо определить точку пересечения $l_{curr}$ и $l$, если окажется, что она принадлежит обоим сегментам, то придется удалить все точки границы, лежащие между рассматриваемыми и добавить точку пересечения, в противном случае нужно перейти к следующей паре отрезков.
\end{enumerate}




\subsection{Результаты работы программы}
\subsubsection{Пример 1}
Возьмём $\alpha = 1$. При таком параметре численно найдем значения неподвижных точек. Построим для наглядности слева от прямой $x = 0$ график функции $f(x) \hm = x \cos(x^2) - \alpha$, его пересечения с осью абсцисс соответствуют неподвижным точкам системы $(S_{+})$. А справа от прямой $x = 0$ изобразим функцию $g(x) = x \cos(x^2) + \alpha$, её обнуления совпадают с особыми точками $(S_{-})$.

\begin{figure}[H]
	\centering{\includegraphics[height=10cm]{P1}}
	\caption{Поиск стационарных точек}
\end{figure}
Чтобы определить характер устойчивости точек, нам понадобится матрица Якоби системы \eqref{eq2}. Примем $f_1(x_1, x_2) = x_2$, а $f_2(x_1, x_2) = -x_1\cos\left(x_1^2\right) - 2x_1x_2 + \alpha$. Тогда 
$$
\mathcal{J}(x_1, x_2) = 
\begin{pmatrix}
	\dfrac{\partial{f}_1}{\partial{x_1}} & \dfrac{\partial{f}_1}{\partial{x_2}} \\[0.75em]
	\dfrac{\partial{f}_2}{\partial{x_1}} & \dfrac{\partial{f}_2}{\partial{x_2}}
\end{pmatrix} =
\begin{pmatrix}
	0 & 1 \\[0.75em]
	2x_1^2\sin(x_1^2) - \cos(x_1^2) -2x_2 & -2x_1
\end{pmatrix}.
$$

Характеристический многочлен $\chi_{J}(\lambda)$ равен
$$\chi_{J}(\lambda) = \lambda^2 + 2x_1\lambda + (2x_2 + \cos(x_1^2) - 2x_1^2\sin(x_1^2)).$$

Воспользуемся тем, что наши стационарные точки располагаются на прямой $x_2 = 0$, в результате получим 
$$\chi_{J}(\lambda) = \lambda^2 + 2x_1\lambda + (\cos(x_1^2) - 2x_1^2\sin(x_1^2)).$$

Исследуем корни характеристического многочлена в особых точках. Возьмём $x_{+}^{(1)} \hm = -1.514$, в таком случае
$$ \chi_{J}(\lambda) = \lambda^2 - 3.028 \lambda - 4.103 \Rightarrow \{ D = (-3.028)^2 + 4 \cdot 4.103 = 25.581 \} \Rightarrow \lambda_{1, 2} = \dfrac{3.028 \pm \sqrt{25.581}}{2}, $$
откуда $\lambda_1 = 4.043$, $\lambda_2 = -1.015$. 

Значит рассматриваемая точка имеет седловой тип, который является неустойчивым. Отметим также, что сумма корней многочлена равна $-2x_1 > 0$ при $x_1 < 0$, следовательно $x_{+}^{(k)}$ может быть лишь неустойчивой. 

Проанализируем первую относительно нуля стационарную точку системы $(S_{-})$, $x_{-}^{(1)} \hm = 1.514$:
$$ \chi_{J}(\lambda) = \lambda^2 + 3.028 \lambda - 4.103 \Rightarrow \{ D = (3.028)^2 + 4 \cdot 4.103 = 25.581 \} \Rightarrow \lambda_{1, 2} = \dfrac{-3.028 \pm \sqrt{25.581}}{2}, $$
откуда $\lambda_1 = 1.015$, $\lambda_2 = -4.043$, то есть $x_{-}^{(1)}$~--- седло (всегда неустойчива). 

На всех графиках, приведённых ниже, синим цветом выделена граница множества достижимости, красным изображается кривая переключений, зелёным отмечаются особые точки для $(S_{+})$, чёрным~--- для $(S_{-})$.

\begin{figure}[H]
\begin{minipage}[h]{0.49\linewidth}
\includegraphics[width=1\linewidth]{P2}
	\caption{$t = 1.2$}
\end{minipage}
\hfill
\begin{minipage}[h]{0.49\linewidth}
\includegraphics[width=1\linewidth]{P3}
	\caption{$t = 1.6$}
\end{minipage}
\end{figure}	

\begin{figure}[H]
	\centering{\includegraphics[height=6cm]{P4}}
	\caption{$t = 1.8$}
\end{figure}

\begin{figure}[H]
	\centering{\includegraphics[height=9cm]{P5}}
	\caption{$t = 2$}
\end{figure}
С увеличением времени $t$ множество достижимости отдаляется от положений равновесия и стремится к бесконечности, причём довольно-таки быстро (за 0.2 секунды убывает более чем на 100, см. рис. 6-7). Тем самым графики подтверждают неустойчивость рассмотренных выше точек. 

Для полноты понимания происходящего покажем, что $x_{+}^{(1)}$ действительно является седлом: выведем на график и сами траектории при $t = 2$ (зелёным цветом изображены траектории, соответствующие $u = -\alpha$, чёрным соответствующие $u = \alpha$).
\begin{figure}[H]
	\centering{\includegraphics[height=9cm]{P6}}
	\caption{Траектории системы при $t = 2$ возле $x_{+}^{(1)}$}
\end{figure}

\subsubsection{Пример 2}
Посмотрим как ведет себя множество достижимости при фиксированном $t = 1$, для различных параметров $\alpha$. 

Ниже проиллюстрируем (рис. 9-13) характер изменения множества достижимости при прохождении $ \alpha$ через значения из $\{0.75, 1.5, 2.25, 3\}$.
\begin{figure}[H]
\begin{minipage}[h]{0.49\linewidth}
\includegraphics[width=1\linewidth]{P7}
	\caption{$\alpha = 0.75$}
\end{minipage}
\hfill
\begin{minipage}[h]{0.49\linewidth}
\includegraphics[width=1\linewidth]{P8}
	\caption{$\alpha = 1.5$}
\end{minipage}
\end{figure}

\begin{figure}[H]
\begin{minipage}[h]{0.49\linewidth}
\includegraphics[width=1\linewidth]{P9}
	\caption{$\alpha = 2.25$}
\end{minipage}
\hfill
\begin{minipage}[h]{0.49\linewidth}
\includegraphics[width=1\linewidth]{P10}
	\caption{$\alpha = 3$}
\end{minipage}
\end{figure}

Представленные рисунки 9-12 свидетельствуют нам о том, что при увеличении значения $\alpha$ множество достижимости также растёт, причём каждое следующее множество содержит предыдущее (см. рис. 13).
\begin{figure}[H]
	\centering{\includegraphics[height=6cm]{P11}}
	\caption{Изменение множества достижимости при фиксированном времени}
\end{figure}

\subsubsection{Пример 3}
При фиксированном $\alpha = 0.2$ продемонстрируем как меняется кривая переключений. 

Ниже (рис. 14-16) приведены графики границы множества достижимости и кривой переключения для $t = 3$, $t = 4$, $t = 5$.

\begin{figure}[H]
\begin{minipage}[h]{0.49\linewidth}
\includegraphics[width=1\linewidth]{P13}
	\caption{$ t = 3$}
\end{minipage}
\hfill
\begin{minipage}[h]{0.49\linewidth}
\includegraphics[width=1\linewidth]{P14}
	\caption{$t = 4$}
\end{minipage}
\end{figure}


\begin{figure}[H]
	\centering{\includegraphics[height=5.92cm]{P15}}
	\caption{$t = 5$}
\end{figure}


\newpage
\begin{thebibliography}{99}
\item Понтрягин~Л.~С., Болтянский~В.~Г., Гамкрелидзе~Р.~В., Мищенко~Е.~Ф. Математическая теория оптимальных процессов. М.: Физматлит, 1961.
\item Чистяков~И.~А. Лекции по оптимальному управлению. 2022-2023.


\end{thebibliography}

\end{document}